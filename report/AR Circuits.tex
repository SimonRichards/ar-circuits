
\documentclass[11pt,journal,compsoc]{IEEEtran}

%\ifCLASSOPTIONcompsoc
  % IEEE Computer Society needs nocompress option
  % requires cite.sty v4.0 or later (November 2003)
  \usepackage[nocompress]{cite}
%\else
  % normal IEEE
  % \usepackage{cite}
%\fi
\ifCLASSINFOpdf
  \usepackage[pdftex]{graphicx}
  % declare the path(s) where your graphic files are
  % \graphicspath{{../pdf/}{../jpeg/}}
  % and their extensions so you won't have to specify these with
  % every instance of \includegraphics
  % \DeclareGraphicsExtensions{.pdf,.jpeg,.png}
\else
  % or other class option (dvipsone, dvipdf, if not using dvips). graphicx
  % will default to the driver specified in the system graphics.cfg if no
  % driver is specified.
  % \usepackage[dvips]{graphicx}
  % declare the path(s) where your graphic files are
  % \graphicspath{{../eps/}}
  % and their extensions so you won't have to specify these with
  % every instance of \includegraphics
  % \DeclareGraphicsExtensions{.eps}
\fi
\usepackage{array}

%\ifCLASSOPTIONcompsoc
\usepackage[tight,normalsize,sf,SF]{subfigure}
%\else
%\usepackage[tight,footnotesize]{subfigure}
%\fi\
%\ifCLASSOPTIONcompsoc
  \usepackage[caption=false,font=normalsize,labelfont=sf,textfont=sf]{subfig}
%\else
%  \usepackage[caption=false,font=footnotesize]{subfig}
%\fi


\usepackage{stfloats}

\hyphenation{op-tical net-works semi-conduc-tor}


\begin{document}
\title{AR Circuits}

\author{Daniel Bentall}% <-this % stops a space

\IEEEcompsoctitleabstractindextext{%
\begin{abstract}
\end{abstract}
}

% make the title area
\maketitle

\IEEEdisplaynotcompsoctitleabstractindextext

\IEEEpeerreviewmaketitle

\section{Introduction}

\IEEEPARstart{D}{ynamic Analysis} bitches \cite{orc13}

\section{Related Work}

In the paper "Augmented Reality for teaching spatial relations"\cite{Maier09} the authors describe their tool Augmented Chemical Reactions, which is aimed at using AR to teach students about how the geometry of molecules effect reactions between them. Molecules are spatially registered to fiducial markers. Marker-cubes can be used to allow the user to view the molecule from any angle. When the markers of two molecules with the ability to bond to each other come into proximity to each other, possible bonds are shown as transparent tubes. The user can then bond the molecules if they wish.

An area for further work not mentioned in the paper could be integrating extra visualisations of the chemical reaction at the visible scale. A common problem with advanced concepts is the difficulty with bridging the gap between abstract concepts and real-world observations. This tool could allow a macroscopic view demonstrating the tangible outputs of a reaction alongside the individual molecule view. Extra data such as temperature and state-changes could be shown to give a more complete understanding of the reaction.

Construct3D is an AR educational tool described in the paper "Mathematics And Geometry Education With Collaborative Augmented Reality"\cite{Kaufmann03}. Using one of a number of suggested hardware set-ups, users create geometric entities such as curves and spheres using a 6 degrees-of-freedom 'stylus' and the Personal Interaction Panel, an AR interaction system\cite{szal97}. With their geometric entities they can then investigate concepts such as geometric sections and vectors. In the more ambitious hardware set-ups, users can walk around and through the geometry they create. The geometry exists on 3D layers, allowing different visibility modes for different users or teaching scenarios. 

Collaboration is emphasised, with the tool featuring shared scene-graphs and AR applications between hardware systems. This advanced form of collaboration with multiple communicating systems is not within the scope of our project, however the simpler collaboration which a single system can support is. While the educational content is not as closely related to our project, the design considerations regarding educational use are highly relevant.

The contexts in which Construct3D would be likely used for were well considered, including the possible participants and the available resources. This means that collaboration at different scales and usage through different mediums such as HMDs and monitors were designed for. An initial trial study showed that students found the tool very useful, and the tool was improved based on feedback. Large-scale studies on the educational benefits were carried out, and are summarised in a paper from 2007\cite{Kaufmann07}. Many improvements were made based on the results of these studies.


\section{Project Goals}

This goals of this project were to use AR to provide a novel electricity education tool. Users build virtual electronic circuits and the states of the electrical components are visually represented. To achieve this, the following general goals were set:

\begin{itemize}
\item Represent various electrical components with some real-world object, each corresponding to a single marker.
\item Allow users to create and destroy connections between components using some intuitive interaction.
\item Visualise this connection.
\item Simulate the current through and voltage drop of each component.
\item Visualise this data.
\end{itemize}

Once the approach for solving these goals had been planned, a set of more specific goals were created:

\begin{itemize}
\item D
\end{itemize}

Students use fiducial markers to interact with their circuits.

\section{System Description}

\section{Discussion}

\subsection{Limitations}

\section{Further Work}

\section{Conclusion}


% trigger a \newpage just before the given reference
% number - used to balance the columns on the last page
% adjust value as needed - may need to be readjusted if
% the document is modified later
%\IEEEtriggeratref{8}
% The "triggered" command can be changed if desired:
%\IEEEtriggercmd{\enlargethispage{-5in}}

% references section

% can use a bibliography generated by BibTeX as a .bbl file
% BibTeX documentation can be easily obtained at:
% http://www.ctan.org/tex-archive/biblio/bibtex/contrib/doc/
% The IEEEtran BibTeX style support page is at:
% http://www.michaelshell.org/tex/ieeetran/bibtex/
\bibliographystyle{IEEEtran}
% argument is your BibTeX string definitions and bibliography database(s)
\bibliography{IEEEabrv,mybib}
%
% <OR> manually copy in the resultant .bbl file
% set second argument of \begin to the number of references
% (used to reserve space for the reference number labels box)

\end{document}